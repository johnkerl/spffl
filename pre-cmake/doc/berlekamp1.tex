%====================================================================
% This is a LaTeX-2e document
%====================================================================

\documentclass[10pt]{article}
\parindent=0pt \parskip=8pt

\usepackage{verbatim}
\usepackage{fullpage}

\usepackage{amssymb,amsmath,amsthm}
\usepackage[mathscr]{eucal}

\usepackage{makeidx}
\numberwithin{equation}{subsection}
%\numberwithin{equation}{section}
%====================================================================
%-------------------------------------------------------------------- 
%
%   isomath.tex  LaTeX macros for math conforming to ISO standards
%
%---------------------------------------- this is a LaTeX-2e document
%====================================================================     

%% Blackboard Bold

\newcommand*{\bbb}[1]{\mathbb{#1}}
\newcommand{\bC}{\bbb{C}}
\newcommand{\bN}{\bbb{N}}
\newcommand{\bQ}{\bbb{Q}}
\newcommand{\bR}{\bbb{R}}
\newcommand{\bZ}{\bbb{Z}}
\newcommand{\bF}{\bbb{F}}
\newcommand{\bFp}{\bbb{F}_p}
\newcommand{\bFpn}{\bbb{F}_{p^n}}
\newcommand{\ii}{\mathbf{i}}
\newcommand{\jj}{\mathbf{j}}
\newcommand{\kk}{\mathbf{k}}

\newcommand{\N}{\bN}
\newcommand{\Q}{\bQ}
\newcommand{\R}{\bR}
\newcommand{\Z}{\bZ}
\newcommand{\ZpZ}{\Z/p\Z}
\newcommand{\ZmZ}{\Z/m\Z}
\newcommand{\C}{\bC}
%\newcommand{\F}{\bF}
\newcommand{\F}{\mathbb{F}}
\newcommand{\Fp}{\bFp}
\newcommand{\Fpn}{\bFpn}

%% Calligraphic

%\newcommand*{\cal}[1]{\mathcal{#1}}
%\newcommand{\cB}{\cal{B}}
%\newcommand{\cC}{\cal{C}}
%\newcommand{\cF}{\cal{F}}
%\newcommand{\cH}{\cal{H}}
%\newcommand{\cK}{\cal{K}}
%\newcommand{\cT}{\cal{T}}

\newcommand{\mC}{\mathcal{C}}
\newcommand{\mT}{\mathcal{T}}

%% Miscellaneous

%yuk -- Kaz's preferences
%\renewcommand{\phi}{\varphi}
%\renewcommand{\theta}{\vartheta}

%% Named Functions and Operators

\newcommand{\id}{\mathrm{id}}
\newcommand{\Id}{\mathrm{Id}}
\newcommand{\rank}{\mathrm{rank}}
\newcommand{\supp}{\mathrm{supp}}
\newcommand{\tr}{\mathrm{tr}}
\newcommand{\Aut}{\mathrm{Aut}}
\newcommand{\Gal}{\mathrm{Gal}}
\newcommand{\Hom}{\mathrm{Hom}}
\newcommand{\im}{\mathrm{im}}
\newcommand{\lcm}{\mathrm{lcm}}
\newcommand{\Tr}{\mathrm{Tr}}
\newcommand{\GL}{\mathrm{GL}}
%\newcommand{\mod}{\textrm{ mod }}

%\DeclareMathOperator{\cod}{Codom}
%\DeclareMathOperator{\dom}{Dom}
%\DeclareMathOperator{\im}{im}
%\DeclareMathOperator{\ran}{Ran}
%\DeclareMathOperator{\supp}{supp}

%\DeclareMathOperator*{\clsp}{\overline{span}}
%\DeclareMathOperator*{\spn}{span}

%% Vectors

\usepackage{amsbsy}
\renewcommand*{\vec}[1]{\boldsymbol{#1}}
\newcommand{\va}{\vec{a}}
\newcommand{\vb}{\vec{b}}
\newcommand{\vc}{\vec{c}}
\newcommand{\vx}{\vec{x}}
\newcommand{\vy}{\vec{y}}
\newcommand{\vz}{\vec{z}}
\newcommand{\vu}{\vec{u}}
\newcommand{\vv}{\vec{v}}
\newcommand{\vw}{\vec{w}}

%% Logical Propositions

\newcommand*{\pro}[1]{\mathsf{#1}}
\newcommand{\pP}{\pro{P}}
\newcommand{\pQ}{\pro{Q}}
\newcommand{\pR}{\pro{R}}

%% Matrices and Tensors

%\DeclareMathAlphabet{\mathsfsl}{OT1}{cmss}{m}{sl}
\newcommand*{\mat}[1]{\mathsfsl{#1}}
\newcommand{\mM}{\mat{M}}
\newcommand{\mD}{\mat{D}}
\newcommand{\mI}{\mat{I}}

%% Special Numbers and Characters

\newcommand{\re}{\mathrm{e}}
\newcommand{\ri}{\mathrm{i}}
\newcommand{\rd}{\mathrm{d}} % differential
\newcommand{\dif}[1]{\,\rd{#1}} % differential
\newcommand{\df}{\dif{f}}
\newcommand{\dx}{\dif{x}}
\newcommand{\dy}{\dif{y}}
\newcommand{\dz}{\dif{z}}
\newcommand{\dt}{\dif{t}}
\newcommand{\rpi}{\pi}       % actually, pi should be upright.

\newcommand{\rE}{\mathsf{E}}
\newcommand{\rF}{\mathsf{F}}
\newcommand{\rI}{\mathsf{I}}
\newcommand{\rK}{\mathsf{K}}
\newcommand{\rL}{\mathsf{L}}
\newcommand{\rM}{\mathsf{M}}
\newcommand{\rN}{\mathsf{N}}
\newcommand{\rP}{\mathsf{P}}
\newcommand{\rT}{\mathsf{T}}
\newcommand{\rU}{\mathsf{U}}
\newcommand{\rV}{\mathsf{V}}
\newcommand{\rW}{\mathsf{W}}
\newcommand{\rX}{\mathsf{X}}
\newcommand{\rY}{\mathsf{Y}}
\newcommand{\rZ}{\mathsf{Z}}


%\newcommand*{\pig}[1]{\langle #1 \rangle}

%====================================================================

%====================================================================
%
%  mathenv.tex  math environments  version 0.00
%
%====================================================================

\usepackage{ifthen}

\theoremstyle{plain}
%\newtheorem{thm}{Theorem}[subsection]
\newtheorem{thm}[equation]{Theorem}
\newtheorem{cor}[equation]{Corollary}
\newtheorem{lem}[equation]{Lemma}
\newtheorem{prop}[equation]{Proposition}
\newtheorem{ax}[equation]{Axiom}
\newtheorem*{thm*}{Theorem}
\newtheorem*{cor*}{Corollary}
\newtheorem*{lem*}{Lemma}
\newtheorem*{prop*}{Proposition}
\newtheorem*{ax*}{Axiom}

\theoremstyle{definition}
\newtheorem{defn}[equation]{Definition}
\newtheorem{problem}{Problem}
\newtheorem*{defn*}{Definition}
\newtheorem*{problem*}{Problem}
\newtheorem{ex}[equation]{Example}
\newtheorem{exs}[equation]{Examples}
\newtheorem*{ex*}{Example}
\newtheorem*{exs*}{Examples}

%\theoremstyle{remark}
\newtheorem{rem}[equation]{Remark}
\newtheorem*{rem*}{Remark}
\newtheorem{aside}[equation]{Aside}
\newtheorem*{aside*}{Aside}
\newtheorem{intuition}[equation]{Intuition}
\newtheorem*{intuition*}{Intuition}
\newtheorem{notn}[equation]{Notation}
\newtheorem*{notn*}{Notation}
\newtheorem{conv}[equation]{Convention}
\newtheorem*{conv*}{Convention}
\newtheorem{interp}[equation]{Interpretation}
\newtheorem*{interp*}{Interpretation}
\newtheorem{mnem}[equation]{Mnemonic}
\newtheorem*{mnem*}{Mnemonic}
\newtheorem{exer}[equation]{Exercise}
\newtheorem*{exer*}{Exercise}
\newtheorem{conjecture}[equation]{Conjecture}
\newtheorem*{conjecture*}{Conjecture}

%\numberwithin{equation}{subsection}
%\numberwithin{equation}{section}

%\newcommand{\beginex}{\begin{ex}$\blacktriangleright$ }
%\newcommand{\mendex}{\hfill$\blacktriangleleft$\end{ex}}
\newcommand{\beginex}{\begin{ex}$\rhd$ }
\newcommand{\mendex}{\hfill$\lhd$\end{ex}}

%--------------------------------------------------------------------
%
%\newenvironment{problem}[1][]
% {\noindent{\bf#1.\ }}
% {}
%
%--------------------------------------------------------------------

\newenvironment{answer}[1][Answer]
  {\begin{proof}[#1]}
  {\renewcommand{\qedsymbol}{}\end{proof}}

\newenvironment{solution}[1][Solution]
  {\begin{proof}[#1]}
  {\renewcommand{\qedsymbol}{\ensuremath{\Diamond}}\end{proof}}

\newcounter{ppart}
\newenvironment{parts}
 {\begin{list}{\textup(\alph{ppart}\textup)\hfill}
   {\usecounter{ppart}
    \settowidth{\labelwidth}{\textup(m\textup)}
    \setlength{\leftmargin}{0cm} 
    \setlength{\rightmargin}{0cm}
    \setlength{\itemindent}{2em}
    \setlength{\labelsep}{\itemindent}
    \addtolength{\labelsep}{-\labelwidth}
    \renewcommand{\makelabel}[1]{\ifthenelse
     {\equal{##1}{}}
     {\textup{(\alph{ppart})}\hfill}
     {\textup{##1}\hfill}}
   }
 }
 {\end{list}}

%\renewcommand{\labelenumi}{(\roman{enumi})}

\newcommand{\secref}[1]{Section~\textup{\ref{#1}}}
\newcommand{\thmref}[1]{Theorem~\textup{\ref{#1}}}
\newcommand{\corref}[1]{Corollary~\textup{\ref{#1}}}
\newcommand{\lemref}[1]{Lemma~\textup{\ref{#1}}}
\newcommand{\propref}[1]{Proposition~\textup{\ref{#1}}}
\newcommand{\defnref}[1]{Definition~\textup{\ref{#1}}}
\newcommand{\remref}[1]{Remark~\textup{\ref{#1}}}
\newcommand{\exref}[1]{Example~\textup{\ref{#1}}}
\newcommand{\exsref}[1]{Examples~\textup{\ref{#1}}}
\newcommand{\axref}[1]{Axiom~\textup{\ref{#1}}}
\newcommand{\itemref}[1]{\textup{{\ref{#1}}}}

% Vim 7.0 syntax highlighting doesn't seem to kick in unless the .tex
% file has at least one begin-end pair in it.  Go figure.
\newenvironment{vim_bug_workaround}
	{}

%====================================================================



\newcommand*{\bfidx}[1]{\index{#1}\textbf{#1}}
\newcommand*{\bfidxtwo}[2]{\textbf{#1}\index{#2}}
\newcommand*{\bfidxaux}[2]{\index{#1}\index{#2}\textbf{#1}}
\newcommand*{\emphidx}[1]{\index{#1}\emph{#1}}
\newcommand*{\plainidx}[1]{\index{#1}#1}

\newcommand{\modp}{\,(\mathrm{mod } \,p)}
\newcommand{\veps}{\varepsilon}
\newcommand{\modfive}{\,(\mathrm{mod } \,5)}
\newcommand{\Fpx}{\F_p^\times}
\newcommand{\Fpd}{\F_{p^d}}
\newcommand{\Fq}{\F_q}


\newcommand*{\rowvectwo}[2]{\begin{pmatrix}#1 & #2 \end{pmatrix}}
\newcommand*{\rowvecthree}[3]{\begin{pmatrix}#1 & #2 & #3\end{pmatrix}}

\newcommand*{\colvecone}[1]{\left(\begin{array}{r} #1 \end{array}\right)}
\newcommand*{\colvectwo}[2]{\left(\begin{array}{r} #1 \\ #2 \end{array}\right)}
\newcommand*{\colvecthree}[3]{\left(\begin{array}{r} #1 \\ #2 \\ #3\end{array}\right)}
\newcommand*{\colvecfour}[4]{\left(\begin{array}{r} #1 \\ #2 \\ #3 \\ #4\end{array}\right)}

\newcommand*{\collvecone}[1]{\left(\begin{array}{l} #1 \end{array}\right)}
\newcommand*{\collvectwo}[2]{\left(\begin{array}{l} #1 \\ #2 \end{array}\right)}
\newcommand*{\collvecthree}[3]{\left(\begin{array}{l} #1 \\ #2 \\ #3\end{array}\right)}
\newcommand*{\collvecfour}[4]{\left(\begin{array}{l} #1 \\ #2 \\ #3 \\ #4\end{array}\right)}

\newcommand*{\colcvecone}[1]{\left(\begin{array}{c} #1 \end{array}\right)}
\newcommand*{\colcvectwo}[2]{\left(\begin{array}{c} #1 \\ #2 \end{array}\right)}
\newcommand*{\colcvecthree}[3]{\left(\begin{array}{c} #1 \\ #2 \\ #3\end{array}\right)}
\newcommand*{\colcvecfour}[4]{\left(\begin{array}{c} #1 \\ #2 \\ #3 \\ #4\end{array}\right)}

\newcommand*{\colrvecone}[1]{\left(\begin{array}{r} #1 \end{array}\right)}
\newcommand*{\colrvectwo}[2]{\left(\begin{array}{r} #1 \\ #2 \end{array}\right)}
\newcommand*{\colrvecthree}[3]{\left(\begin{array}{r} #1 \\ #2 \\ #3\end{array}\right)}
\newcommand*{\colrvecfour}[4]{\left(\begin{array}{r} #1 \\ #2 \\ #3 \\ #4\end{array}\right)}


%% ----------------------------------------------------------------
% Parenthesized matrix with left alignment.
\newcommand*{\plmatrix}[1]{
	\left(\begin{array}{llllllllllll}
		#1
	\end{array}\right)
}

% Parenthesized matrix with centered alignment.
\newcommand*{\pcmatrix}[1]{
	\left(\begin{array}{cccccccccccc}
		#1
	\end{array}\right)
}

% Parenthesized matrix with right alignment.
\newcommand*{\prmatrix}[1]{
	\left(\begin{array}{rrrrrrrrrrrr}
		#1
	\end{array}\right)
}

%% ----------------------------------------------------------------
% Bracketed matrix with left alignment.
\newcommand*{\blmatrix}[1]{
	\left[\begin{array}{llllllllllll}
		#1
	\end{array}\right]
}

% Bracketed matrix with centered alignment.
\newcommand*{\bcmatrix}[1]{
	\left[\begin{array}{cccccccccccc}
		#1
	\end{array}\right]
}

% Bracketed matrix with right alignment.
\newcommand*{\brmatrix}[1]{
	\left[\begin{array}{rrrrrrrrrrrr}
		#1
	\end{array}\right]
}

%% ----------------------------------------------------------------
% Unbracketed matrix with left alignment.
\newcommand*{\nlmatrix}[1]{
	\begin{array}{llllllllllll}
		#1
	\end{array}
}

% Unbracketed matrix with centered alignment.
\newcommand*{\ncmatrix}[1]{
	\begin{array}{cccccccccccc}
		#1
	\end{array}
}

% Unbracketed matrix with right alignment.
\newcommand*{\nrmatrix}[1]{
	\begin{array}{rrrrrrrrrrrr}
		#1
	\end{array}
}

\usepackage{graphicx}
\usepackage{psfrag}

\makeindex

\usepackage{hyperref}
\usepackage{ifpdf}

\ifpdf
\pdfinfo{
/Author (John Kerl)
/Title  (The Berlekamp algorithm)
}
\fi

%--------------------------------------------------------------------
\begin{document}
\title{The Berlekamp algorithm}
\author{John Kerl}
\date{\today}

\maketitle
\begin{abstract}
We describe Berlekamp's algorithm for efficient factorization of polynomials
over finite fields $\Fq$.  For simplicity, we restrict the examples to base
fields $\Fp$ where $p$ is prime.  Xref to Lidl and Niederreiter.
\end{abstract}

\addcontentsline{toc}{section}{Contents}
\tableofcontents

%% ================================================================
\newpage
\section{Examples}

To fix ideas, we list fourth-degree polynomials over $\F_2$, along with their
factorizations.

For example, $x^4 + 1$ is a fourth-degree polynomial with coefficients 0 or 1,
with coefficient arithmetic done mod 2.  This factors as $(x^2+1)^2$, which in
turn factors as $(x+1)^4$.  So, $x^4+1$ is \emphidx{reducible}.

By contrast, $x^4+x+1$ has no non-trivial factors; we say that it is
\emphidx{reducible}.  How can I be so sure of that statement?  Well, since
degree four is small, we can do an explicit search.  For larger degree's,
Berlekamp's algorithm is a powerful way to factor polynomials over $\Fq$.  Note
that we don't have such a powerful algorithm for polynomials with coefficients
in $\Q$, or for integers.  (Some cryptosystems, such as RSA, are in fact
designed around the difficulty of integer factorization.)

Let's see why $x^4+x+1$ is irreducible over $\F_2$.  If it has a non-trivial
factor, i.e. a factor with degree greater than or equal to one, one factor must
have degree 1, 2, or 3.  There are two degree-one polynomials over $\F_2$:  $x$
and $x+1$.  There are four degree-two polynomials: $x^2$, $x^2+1$, $x^2+x$, and
$x^2+x+1$.  There are eight degree-three polynomials, $x^3$ through
$x^3+x^2+x+1$.  So, all we need to do is use long division, dividing $x^4+x+1$
by each of these fourteen possible divisors.  If none of them goes into
$x^4+x+1$ with a zero remainder, then $x^4+x+1$ is irreducible.  (This is
similar to trial division for integers:  given an integer $n$, the most naive
factorization algorithm is to try dividing $n$ by every integer $2 \le d < n$.
Improvements are possible, of course, but that's the very most naive method.)

For shorthand, we'll often write polynomials with their coefficients only, in
descending order of degrees of terms.  For example, $x^4+1 = 10001$ and
$x^4+x+1=10011$.

%Here are all the fourth-degree polynomials over $\F_2$ along
%with their factorizations (which you can check by multiplying them back out):
%\begin{align*}
%	10000 &= 10^2   & 10100 &= 11^2\cdot 10^2   & 11000 &= 11\cdot 10^3   & 11100 &= 111 \cdot 10^2   \\
%	10001 &= 11^2   & 10101 &= xxx   & 11001 &= xxx   & 11101 &= xxx   \\
%	10010 &= xxx   & 10110 &= xxx   & 11010 &= xxx   & 11110 &= xxx   \\
%	10011 &= xxx   & 10111 &= xxx   & 11011 &= xxx   & 11111 &= 11111   \\
%\end{align*}

%% ================================================================
\newpage
\section{Squarefree preparation}

xxx show where the algorithm requires that.

Monicity.

$g = \gcd(f,f')$

If $g$ has degree 0, input is squarefree, and is ready for Berlekamp.

If $f' = 0$:  the input is a perfect $p$th power.  Get the $p$th root.
Maybe recurse in case the root is itself a $p$th power.

Else, recurse on $g$ and $f/g$.

%% ================================================================
\newpage
\section{Berlekamp's algorithm}

%% ================================================================
\newpage
\section{Example}

Given squarefree $f(x)$, we want to find polynomials $h(x)$
such that $h^q \equiv h \mod f$.  Take $f = x^5+x^4+1 =
110001 = 111 \cdot 1011$.  The degree of $f$ is 5.  By
explicit search, we can find the following polynomials $h$
of degree $< 5$ such that $h^2 \equiv h \mod f$ (Lidl and
Niederreiter call these ``f-reducing polynomials''): 0, 1,
11100, 11101.

To avoid having to do a search, we use linear algebra instead.
\begin{align*}
	f &= x^5+x^4+1 = 110001 \\
	x^0 &\equiv 00001 \mod f \\
	x^2 &\equiv 00100 \mod f \\
	x^4 &\equiv 10000 \mod f \\
	x^6 &\equiv 10011 \mod f \\
	x^8 &\equiv 11111 \mod f.
\end{align*}
\begin{align*}
h   &= a_4 x^4 + a_3 x^3 + a_2 x^2 + a_1 x   + a_0 \\
h^2 &= a_4 x^8 + a_3 x^6 + a_2 x^4 + a_1 x^2 + a_0 \\
    &= a_4(x^4+x^3+x^2+x+1) + a_3(x^4+x+1) + a_2(x^4) + a_1(x^2) + a_0(1) \\
%
%
\bcmatrix{
	a_4 \\
	a_3 \\
	a_2 \\
	a_1 \\
	a_0 \\
}
&=
\bcmatrix{
1 & 1 & 1 & 0 & 0 \\
1 & 0 & 0 & 0 & 0 \\
1 & 0 & 0 & 1 & 0 \\
1 & 1 & 0 & 0 & 0 \\
1 & 1 & 0 & 0 & 1 \\
}
\cdot
\bcmatrix{
	a_4 \\
	a_3 \\
	a_2 \\
	a_1 \\
	a_0 \\
} \\
%
\bcmatrix{
	0 \\
	0 \\
	0 \\
	0 \\
	0 \\
}
&=
\bcmatrix{
0 & 1 & 1 & 0 & 0 \\
1 & 1 & 0 & 0 & 0 \\
1 & 0 & 1 & 1 & 0 \\
1 & 1 & 0 & 1 & 0 \\
1 & 1 & 0 & 0 & 0 \\
}
\cdot
\bcmatrix{
	a_4 \\
	a_3 \\
	a_2 \\
	a_1 \\
	a_0 \\
}.
\end{align*}
Call the matrix B.  Its $(n-1-j)$th column is $x^{jq} \mod f$.  Put $B-I$ in
row-echelon form to obtain
\begin{align*}
\bcmatrix{
	1 & 0 & 1 & 0 & 0 \\
	0 & 1 & 1 & 0 & 0 \\
	0 & 0 & 0 & 1 & 0 \\
	0 & 0 & 0 & 0 & 0 \\
	0 & 0 & 0 & 0 & 0 \\
	}
\end{align*}
with kernel basis
\begin{align*}
	(1,1,1,0,0), \qquad (0,0,0,0,1).
\end{align*}
These are $h_1 = 11100$ and $h_2 = 1$, respectively.  Compute $\gcd(f, h_1) = 111$ and
$\gcd(f, h_1+1) = 1011$ to obtain non-trivial factors of $f$.

%static void fp_poly_berlekamp(
%	fp_poly_t   f,
%	tfacinfo<fp_poly_t> & rfinfo,
%	int recurse)
%{
%	int n  = f.find_degree();
%	int p = f.get_char();
%	intmod_t zero(0, p);
%	intmod_t one(1, p);
%	fp_poly_t x(one, zero);
%	fp_poly_t xp;
%	fp_poly_t xpi = one;
%	fp_poly_t f1, f2;
%	int i, j, row, rank, dimker;
%
%	xp = x;
%	for (int i = 1; i < p; i++)
%		xp = (xp * x) % f;
%
%	tmatrix<intmod_t> BI(n, n);
%
%	if (n < 2) {
%		rfinfo.insert_factor(f);
%		return;
%	}
%
%	// Populate the B matrix.
%	for (j = 0; j < n; j++) {
%		for (i = 0; i < n; i++) {
%			BI[n-1-i][n-1-j] = xpi.get_coeff(i);
%		}
%		xpi = (xpi * xp) % f;
%	}
%
%	// Form B - I.
%	for (i = 0; i < n; i++)
%		BI[i][i] -= one;
%
%	BI.row_echelon_form();
%
%	rank = BI.get_rank_rr();
%	dimker = n - rank;
%
%	if (dimker == 1) {
%		rfinfo.insert_factor(f);
%		return;
%	}
%
%	// Find a basis for the nullspace of B - I.
%	tmatrix<intmod_t> nullspace_basis;
%	if (!BI.get_kernel_basis(nullspace_basis, zero, one)) {
%		std::cerr << "Coding error: file "
%			<< __FILE__ << " line "
%			<< __LINE__ << "\n";
%		exit(1);
%	}
%	if (nullspace_basis.get_num_rows() != dimker) {
%		std::cerr << "Coding error: file "
%			<< __FILE__ << " line "
%			<< __LINE__ << "\n";
%		exit(1);
%	}
%
%	int got_it = 0;
%	for (row = 0; row <  dimker && !got_it; row++) {
%		fp_poly_t h = fp_poly_from_vector(nullspace_basis[row], n);
%		if (h == 1)
%			continue;
%
%		for (int c = 0; c < p; c++) {
%			fp_poly_t hc = h - f.prime_sfld_elt(c);
%			f1 = f.gcd(hc);
%			int f1d = f1.find_degree();
%			int fd  = f.find_degree();
%			if ((0 < f1d) && (f1d < fd)) {
%				got_it = 1;
%				break;
%			}
%		}
%	}
%
%	if (!got_it) {
%		// No non-trivial factors found.
%		std::cerr << "Coding error: file "
%			<< __FILE__ << " line "
%			<< __LINE__ << "\n";
%		exit(1);
%	}
%
%	f2 = f / f1;
%
%	// Input was already made monic, so these factors should
%	// be as well.
%	f1 /= f1.get_coeff(f1.find_degree());
%	f2 /= f2.get_coeff(f2.find_degree());
%
%	// The nullity of B-I is the number of irreducible
%	// factors of r.  If the nullity is 2, we have a
%	// pair of factors which are both irreducible and
%	// so we don't need to recurse.
%	if (dimker == 2) {
%		rfinfo.insert_factor(f1);
%		rfinfo.insert_factor(f2);
%	}
%	else if (!recurse) {
%		rfinfo.insert_factor(f1);
%		rfinfo.insert_factor(f2);
%	}
%	else {
%		fp_poly_pre_berlekamp(f1, rfinfo, recurse);
%		fp_poly_pre_berlekamp(f2, rfinfo, recurse);
%	}
%}




%static void f2_poly_berlekamp(
%	f2_poly_t   f,
%	tfacinfo<f2_poly_t> & rfinfo,
%	int recurse)
%{
%	int n  = f.find_degree();
%	f2_poly_t x(1, 0);
%	f2_poly_t x2 = (x * x) % f;
%	f2_poly_t x2i(1);
%	int i, j, row, rank, dimker;
%	//bit_t zero(0);
%	//bit_t one(1);
%
%	tmatrix<bit_t> BI(n, n);
%
%	if (n < 2) {
%		rfinfo.insert_factor(f);
%		return;
%	}
%
%	// Populate the B matrix.
%	for (j = 0; j < n; j++) {
%		for (i = 0; i < n; i++) {
%			BI[n-1-i][n-1-j] = bit_t(x2i.bit_at(i));
%		}
%		x2i = (x2i * x2) % f;
%	}
%
%	// Form B - I.
%	for (i = 0; i < n; i++) {
%		BI[i][i] = BI[i][i] - one;
%	}
%
%	BI.row_echelon_form();
%
%	rank = BI.get_rank_rr();
%	dimker = n - rank;
%
%	if (dimker == 1) {
%		rfinfo.insert_factor(f);
%		return;
%	}
%
%	// Find a basis for the nullspace of B - I.
%	tmatrix<bit_t> nullspace_basis;
%
%	int got = BI.get_kernel_basis(nullspace_basis, zero, one);
%
%	if (!got) {
%		std::cerr << "Coding error: file "
%			<< __FILE__ << " line "
%			<< __LINE__ << "\n";
%		exit(1);
%	}
%	if (nullspace_basis.get_num_rows() != dimker) {
%		std::cerr << "Coding error: file "
%			<< __FILE__ << " line "
%			<< __LINE__ << "\n";
%		exit(1);
%	}
%
%	// For each h in the nullspace basis, form
%	//   f1 = gcd(f, h)
%	//   f2 = gcd(f, h-1)
%	// Now, the polynomial h=1 is always in the nullspace, in which case
%	// f1 = 1 and f2 = f, which results in a trivial factorization.  Any
%	// of the other h's will work fine, producing a non-trivial
%	// factorization of f into two factors.  (Note that either or both
%	// of them may be reducible, in which we'll need to apply this
%	// algorithm recursively until we're down to irreducible factors.)
%	// Here, for the sake of illustration, is what happens with all the
%	// h's, even though we need only one of them (with f = 703):
%
%	//h=001
%	//  f1: 001
%	//  f2: 70e = 2 3 7 b d
%	//h=102     = 2 3 b d
%	//  f1: 102 = 2 3 b d
%	//  f2: 007 = 7
%	//h=284     = 2 2 7 3b
%	//  f1: 00e = 2 7
%	//  f2: 081 = 3 b d
%	//h=0e8     = 2 2 2 3 b
%	//  f1: 03a = 2 3 b
%	//  f2: 023 = 7 d
%	//h=310     = 2 2 2 2 7 b
%	//  f1: 062 = 2 7 b
%	//  f2: 017 = 3 d
%
%	for (row = 0; row <  dimker; row++) {
%		f2_poly_t h, hc;
%		h = f2_poly_from_vector(nullspace_basis[row], n);
%		hc = h + f2_poly_t(1);
%
%		f2_poly_t check1 = (h  * h)  % f;
%		f2_poly_t check2 = (hc * hc) % f;
%		if ((h != check1) || (hc != check2)) {
%			std::cerr << "Coding error: file "
%				<< __FILE__ << " line "
%				<< __LINE__ << "\n";
%			std::cerr << "  h  = " << h
%				<< "  h^2  = " << check1 << "\n";
%			std::cerr << "  hc = " << hc
%				<< "  hc^2 = " << check2 << "\n";
%			exit(1);
%		}
%
%		f2_poly_t f1 = f.gcd(h);
%		f2_poly_t f2 = f.gcd(hc);
%
%		if ((f1 == 1) || (f2 == 1))
%			continue;
%
%		// The nullity of B-I is the number of irreducible
%		// factors of f.  If the nullity is 2, we have a
%		// pair of factors which are both irreducible and
%		// so we don't need to recurse.
%		if (dimker == 2) {
%			rfinfo.insert_factor(f1);
%			rfinfo.insert_factor(f2);
%		}
%		else if (!recurse) {
%			rfinfo.insert_factor(f1);
%			rfinfo.insert_factor(f2);
%		}
%		else {
%			f2_poly_pre_berlekamp(f1, rfinfo, recurse);
%			f2_poly_pre_berlekamp(f2, rfinfo, recurse);
%		}
%		return;
%	}
%	std::cerr << "Coding error: file "
%		<< __FILE__ << " line "
%		<< __LINE__ << "\n";
%	exit(1);
%}

%% ================================================================
\newpage
\section{Complexity analysis}

\section*{}
\addcontentsline{toc}{section}{References}
xxx

\section*{}
\addcontentsline{toc}{section}{Index}
\printindex

\end{document}
